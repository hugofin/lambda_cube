    
\documentclass[11pt]{article}
\usepackage{times}
    \usepackage{fullpage}
    
    \title{ 3D Animation of Barendregt's Lambda Cube}
    \author{ Hugo Findlay - 2554911f }

    \begin{document}
    \maketitle
    
    
     

\section{Status report}

\subsection{Proposal}\label{proposal}

\subsubsection{Motivation}\label{motivation}

The Lambda cube is a fascinating visualisation of how a combination of 
three different type systems can be added on to the simply typed lambda calculus.
Despite being such an interesting concept, there is a relative lack of 
available learning resources online.  The motivation of this project is to 
create a website that uses a 3D animated cube to teach you more about how
the type systems interact at each node.

\subsubsection{Aims}\label{aims}

The aim of this project is to create a website that is a more effective 
introduction into the lambda cube than what currently exists on the 
internet now.  This will be achieved by using the interactive medium
to its fullest extent, letting the site's user explore the cube at their
own pace, in whichever order they see fit, as well as using the 3D space to
help show how the spatial relations of each calculus.

\subsection{Progress}\label{progress}

\begin{itemize}
    \tightlist
    \item Learned the fundamentals of the Elm programming language, as well as
    the GLSL and MathML languages and the plugins used to integrate them with Elm
    \item Created an animated, navigable rendition of A cube
    \item Designed and programmed the web interface to host the animation and 
    information about each node
    \item Reading about the untyped lambda calculus, as well as on several of the
    type systems so that I will be able to write the explanation at each node.

\end{itemize}

\subsection{Problems and risks}\label{problems-and-risks}

\subsubsection{Problems}\label{problems}

\begin{itemize}
    \tightlist
\item I had a small setback when I initally started to create the animations
using an animation library for haskell, but I realised that this was a bad 
way to go about the probem, so switched to using elm + GLSL.
\item There is also sometimes a realtive lack of information available about
specific calculi in the cube, simply because there has been less research into
their use.  This can sometimes make reading about them a difficult process.

\end{itemize}

\subsubsection{Risks}\label{risks}

\begin{itemize}
    \tightlist
\item A potential risk I could have with testing the efficacy of the website is
finding a suitable trial group, as they should be familiar with the concepts, but not
have any in depth knowledge about lambda calculus \textbf{Mitigation}: I have been reaching 
out to people who I know took the programming languages course in their third year to ask if 
they would be available for the trial
\item It would be difficult to gather quantitative data proving that the website is
an improvement relative to the currently available learning resources

\end{itemize}

\section{Plan}\label{plan}

\begin{itemize}
    \tightlist
    \item
      Week 1-3: Have a finished first version of the website
    \item
      Week 4: Run the user trials and gather feedback
    \item
      Week 5-7: implement the feedback provided by the trials group, deliver the first draft of the writeup to my supervisor
    \item
      Week 8-10: implement my supervisor's writeup feedback
    \end{itemize}
    
    
\subsection{Ethics and data}\label{ethics}
\emph
Options for ethics:

This project will involve tests with human users.  These will be user studies
using standard hardware, and require no personally identifiable information to be captured.
I have verified that the ethics checklist will apply to any evaluation I need to do. I will sign and complete the checklist.


\end{document}
