% REMEMBER: You must not plagiarise anything in your report. Be extremely careful.

\documentclass{l4proj}

    
%
% put any additional packages here
%

\begin{document}

%==============================================================================
%% METADATA
\title{3D Animation of Barendregt's Lambda Cube}
\author{Hugo Findlay}
\date{September 25th, 2023}

\maketitle

%==============================================================================
%% ABSTRACT
\begin{abstract}
    The goal of this project was to create a website with an animated three dimensional representation of the lambda cube, that could be navigated by a user to help them learn about it.
    
\end{abstract}

%==============================================================================

% EDUCATION REUSE CONSENT FORM
% If you consent to your project being shown to future students for educational purposes
% then insert your name and the date below to  sign the education use form that appears in the front of the document. 
% You must explicitly give consent if you wish to do so.
% If you sign, your project may be included in the Hall of Fame if it scores particularly highly.
%
% Please note that you are under no obligation to sign 
% this declaration, but doing so would help future students.
%
\def\consentname {Hugo Findlay} % your full name
\def\consentdate {14 February 2024} % the date you agree
%
\educationalconsent


%==============================================================================
\tableofcontents

%==============================================================================
%% Notes on formatting
%==============================================================================
% The first page, abstract and table of contents are numbered using Roman numerals and are not
% included in the page count. 
%
% From now on pages are numbered
% using Arabic numerals. Therefore, immediately after the first call to \chapter we need the call
% \pagenumbering{arabic} and this should be called once only in the document. 
%
% Do not alter the bibliography style.
%
% The first Chapter should then be on page 1. You are allowed 40 pages for a 40 credit project and 30 pages for a 
% 20 credit report. This includes everything numbered in Arabic numerals (excluding front matter) up
% to but excluding the appendices and bibliography.
%
% You must not alter text size (it is currently 10pt) or alter margins or spacing.
%
%
%==================================================================================================================================
%
% IMPORTANT
% The chapter headings here are **suggestions**. You don't have to follow this model if
% it doesn't fit your project. Every project should have an introduction and conclusion,
% however. 
%
%==================================================================================================================================
\chapter{Introduction}

% reset page numbering. Don't remove this!
\pagenumbering{arabic} 

This chapter talks about the motivation behind the project and its aims.  There then follows an outline of how the rest of the paper is structured

\section{Motivation}

As of this April, it will have been thirty-three years since the Lambda Cube was first described by Henk Barendregt.  However, it has been an existence marked by obscurity; having little presence on the internet, and rarely being fully depicted.  The motivation of this project was to bring the cube to light and make it easier for students to learn about it.

The web format offers the capability to represent the cube as a truly three dimensional experience. This means we are able to show the interconnected nature of the systems which make up the lambda cube in a more relatable and visceral way than a simple paper explanation.  The interactive nature of the website will offer the user to learn by doing rather than just by reading.

There is also a difficulty in locating sources about the systems which comprise the cube.  Much of this is because the field of lambda calculus is diverse and old, and interacts with many other fields of mathematics and computing.  A result of this is a wide range of sources, with many using different terminology to describe identical concepts.



\section{Aims}

The aims which were identified before the start of the project were: 
\begin{itemize}
    \item
    \textbf{to make learning about the lambda cube more accessible} by giving concise, consistent and relevant information about each system.  To do this I would have to write about the theory behind each type system as well as explaining the differences in the syntax, beta reduction rules and typing rules in each system or corner of the cube.

    \item
    \textbf{to properly take advantage of the interactivity of the web medium}, as the common explanations around today are mostly in the same format as when the theory was first introduced in the early 1990's.  By allowing the website's user to navigate around a three dimensional representation of the cube, they should be better able to grasp the interconnectedness of the systems, and be more engaged by the learning process by choosing the order in which they learn about the systems.

    \item
    \textbf{to collate useful sources about the systems comprising the lambda cube} as there is not a commonly or easily available list of reliable textbooks or papers about each system.  In order for the website to be a useful educational resource, I would have to find appropriate and up to date sources for each combination of type systems in the cube, and display them in a clear and consistent manner.
\end{itemize}

\section{Outline}

The remainder of this paper explores and evaluates how I approached these aims, and how successful I was in achieving them.

\begin{itemize}
    \item
    \textbf{Prior Art} focuses on the lambda cube's extant representations online and in paper,  evaluating their merits and considering their weaknesses
    \item
    \textbf{Requirements} talks about the specific  of the website at the start of the project and how they were derived
    \item
    \textbf{Design} details the research I undertook to select the technology I used and to find the sources
    \item
    \textbf{Implementation} covers issues which arose during the development process
    \item
    \textbf{Evaluation} discusses the user studies, their results and the actions I took to implement the feedback
    \item
    \textbf{Conclusion} retrospects the project as a whole and considers how effectively I achieved the aims outlined in the first chapter
    
\end{itemize}

%==================================================================================================================================
\chapter{Prior Art}

The lambda cube 1991 paper

The Wikipedia page

Lambda Cube Unboxed



%==================================================================================================================================
\chapter{Requirements}
What is the problem that you want to solve, and how did you arrive at it?Make it clear how you derived the constrained form of your problem via a clear and logical process. 
\section{Problem Specification}
\section{User Stories}
\section{Functional Requirements}
\section{Non-Functional Requirements}
\section{Discussion}
%==================================================================================================================================
\chapter{Design}
How is this problem to be approached, without reference to specific implementation 
details? 

Design should cover the abstract design in such a way that someone else might be able to do what you did, but with a different language or library or tool.

\section{Web Framework}
\section{Animation Framework}
\section{Maths Framework}

\section{User Interface}

\section{Research}

%==================================================================================================================================
\chapter{Implementation}
What did you do to implement this idea, and what technical achievements did you make? You can't talk about everything. Cover the high level first, then cover important, relevant or impressive details.







\section{Using WebGl}

\section{Using Elm}

\section{LaTeX2Elm Translator}

\section{Handover}

\section{Summary}


%==================================================================================================================================
\chapter{Evaluation} 
How good is your solution? How well did you solve the general problem, and what evidence do you have to support that?


\begin{itemize}
    \item
        Ask specific questions that address the general problem.
    \item
        Answer them with precise evidence (graphs, numbers, statistical
        analysis, qualitative analysis).
    \item
        Be fair and be scientific.
    \item
        The key thing is to show that you know how to evaluate your work, not
        that your work is the most amazing product ever.
\end{itemize}
\section{Experiment Methodology}
\section{Study Results}

\section{Conclusions}
Make sure you present your evidence well. Use appropriate visualisations, reporting techniques and statistical analysis, as appropriate.

If you visualise, follow the basic rules, as illustrated in Figure \ref{fig:boxplot}:
\begin{itemize}
\item Label everything correctly (axis, title, units).
\item Caption thoroughly.
\item Reference in text.
\item \textbf{Include appropriate display of uncertainty (e.g. error bars, Box plot)}
\item Minimize clutter.
\end{itemize}

%==================================================================================================================================
\chapter{Conclusion}    
Summarise the whole project for a lazy reader who didn't read the rest (e.g. a prize-awarding committee).

\begin{itemize}
    \item
        Summarise briefly and fairly.
    \item
        You should be addressing the general problem you introduced in the
        Introduction.        
    \item
        Include summary of concrete results (``the new compiler ran 2x
        faster'')
    \item
        Indicate what future work could be done, but remember: \textbf{you
        won't get credit for things you haven't done}.
\end{itemize}

\section{Summary}

\section{Reflection}



\section{Future Work}

%==================================================================================================================================
%
% 
%==================================================================================================================================
%  APPENDICES  

\begin{appendices}

\chapter{Appendices}

Typical inclusions in the appendices are:

\begin{itemize}
\item
  Copies of ethics approvals (required if obtained)
\item
  Copies of questionnaires etc. used to gather data from subjects.
\item
  Extensive tables or figures that are too bulky to fit in the main body of
  the report, particularly ones that are repetitive and summarised in the body.

\item Outline of the source code (e.g. directory structure), or other architecture documentation like class diagrams.

\item User manuals, and any guides to starting/running the software.

\end{itemize}

\textbf{Don't include your source code in the appendices}. It will be
submitted separately.

\section{Task Sheet for User Evaluation}

\section{User Evaluation Results}

\end{appendices}

%==================================================================================================================================
%   BIBLIOGRAPHY   

% The bibliography style is abbrvnat
% The bibliography always appears last, after the appendices.

\bibliographystyle{abbrvnat}

\bibliography{l4proj}

\end{document}
